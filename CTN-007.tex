\documentclass[OPS,lsstdraft,authoryear,toc]{lsstdoc}
% GENERATED FILE -- edit this in the Makefile
\newcommand{\lsstDocType}{CTN}
\newcommand{\lsstDocNum}{007}
\newcommand{\vcsRevision}{d542820-dirty}
\newcommand{\vcsDate}{2025-10-27}


% Package imports go here.

% Local commands go here.

%If you want glossaries
%\input{aglossary.tex}
%\makeglossaries

\title{The LSST Camera commissioning}

% This can write metadata into the PDF.
% Update keywords and author information as necessary.
\hypersetup{
    pdftitle={The LSST Camera commissioning},
    pdfauthor={YousukeUtsumi},
    pdfkeywords={}
}

% Optional subtitle
% \setDocSubtitle{A subtitle}

% put your name in authors.yaml. I will occasionally update it using the official way or ping me
%% DO NOT EDIT THIS FILE. IT IS GENERATED FROM db2authors.py"
%% Regenerate using:
%%    python $LSST_TEXMF_DIR/bin/db2authors.py -m lsstdoc

\author{
LSST~Camera~Team,
and
Yousuke~Utsumi
}


\setDocRef{CTN-007}
\setDocUpstreamLocation{\url{https://github.com/lsst/ctn-007}}

\date{\vcsDate}

% Optional: name of the document's curator
% \setDocCurator{The Curator of this Document}

\setDocAbstract{%
This technote describes what we found during the LSST Camera commissioning.
}

% Change history defined here.
% Order: oldest first.
% Fields: VERSION, DATE, DESCRIPTION, OWNER NAME.
% See LPM-51 for version number policy.
\setDocChangeRecord{%
  \addtohist{1}{YYYY-MM-DD}{Unreleased.}{Utsumi}
}


\begin{document}

% Create the title page.
\maketitle
% Frequently for a technote we do not want a title page  uncomment this to remove the title page and changelog.
% use \mkshorttitle to remove the extra pages

% ADD CONTENT HERE
% You can also use the \input command to include several content files.

\appendix

\section{Introduction}

\section{Findings}
%% Let's make a a folder for each topic!!
- Sequencer change -- YU

- Bright star test -- SB

- On-sky persistence test -- just showing a dark image after a bright star to say no persitence

- Correlated noise -- Pierre Astier or Pierre Antilogus

- PTC -- Pierre Astier

- R30/S12 and R23/Reb10 -- I will try to collect all the information from Slack

- Temp sensitivity issue -- we can put some episode and what we have done in this document, maybe Theo can describe a few episode?

- Phosphorescence study with u and g band exposures -- SB

- IdleClear re-implementation -- SM

- Pinhole repeatability tests -- YU

- FCS commissioning performance -- Pierre Antilogus, Alexandre etc...

- OCS integration kfanning


\section{Acknowledgements}

This material is based upon work supported in part by the National Science Foundation through Cooperative Agreements AST-1258333 and AST-2241526 and Cooperative Support Agreements AST-1202910 and AST-2211468 managed by the Association of Universities for Research in Astronomy (AURA), and the Department of Energy under Contract No.\ DE-AC02-76SF00515 with the SLAC National Accelerator Laboratory managed by Stanford University.
Additional Rubin Observatory funding comes from private donations, grants to universities, and in-kind support from LSST-DA Institutional Members.

% Include all the relevant bib files.
% https://lsst-texmf.lsst.io/lsstdoc.html#bibliographies
\section{References} \label{sec:bib}
\renewcommand{\refname}{} % Suppress default Bibliography section
\bibliography{local,lsst,lsst-dm,refs_ads,refs,books}

% Make sure lsst-texmf/bin/generateAcronyms.py is in your path
\section{Acronyms} \label{sec:acronyms}
\addtocounter{table}{-1}
\begin{longtable}{p{0.145\textwidth}p{0.8\textwidth}}\hline
\textbf{Acronym} & \textbf{Description}  \\\hline

AST & NSF Division of Astronomical Sciences \\\hline
AURA & Association of Universities for Research in Astronomy \\\hline
CTN & Camera Technical Note \\\hline
DE-AC02 & Department of Energy contract number prefix \\\hline
FCS & Filter Changer System \\\hline
LSST & Legacy Survey of Space and Time (formerly Large Synoptic Survey Telescope) \\\hline
LSST-DA & LSST Discovery Alliance \\\hline
OCS & Observatory Control System \\\hline
PTC & Photon Transfer Curve \\\hline
SB & Surface Brightness \\\hline
SLAC & SLAC National Accelerator Laboratory \\\hline
SM & Sparse Mode \\\hline
\end{longtable}

% If you want glossary uncomment below -- comment out the two lines above
%\printglossaries





\end{document}
